\documentclass{scrartcl}
\usepackage{scrhack}

\subject{Bericht}
\titlehead{%
  Humboldt-Universität zu Berlin\\
  Mathematisch-Naturwissenschaftliche Fakultät\\
  Institut für Mathematik
}
\title{Annäherungen von \(\pi\)}
\author{
  Eingereicht von M. van Straten und P. Merz
}
\date{\today}

\usepackage[ngerman]{babel}
\usepackage{amsmath}
\usepackage{amssymb}
\usepackage{amsthm}
\usepackage{microtype}
\usepackage{hyperref}
\usepackage{graphicx}
\usepackage{booktabs}
\usepackage[
backend=biber,
style=alphabetic,
sorting=ynt
]{biblatex}
\usepackage{pdfpages}

\addbibresource{./map.bib}

\newtheorem{definition}{Definition}

\begin{document}
\maketitle
\cleardoublepage{}
\tableofcontents
\cleardoublepage{}

\section{Motivation und Einleitung}

Die Konstante \(\pi\) hat eine natürliche Definition in der euklidischen
Geometrie als das Verhältnis zwischen dem Umfang und dem Durchmesser eines
Kreises.
Sie ist von fundamentaler Bedeutung in vielen Bereichen der Mathematik und
Physik.
Beispielsweise tritt \(\pi\) im Gaußschen Integral auf, bei den komplexen
Einheitswurzeln und in der Cauchy-Verteilung in der Wahrscheinlichkeitstheorie.
Die Relevanz von \(\pi\) erstreckt sich dabei über zahlreiche Disziplinen
hinweg, was sie zu einer der wichtigsten Konstanten in der Wissenschaft macht.

Historisch gesehen haben sich Mathematiker stets bemüht, \(\pi\) immer genauer
zu approximieren.
In der Antike reichte die Genauigkeit solcher Annäherungen oft nur bis zur
zweiten Nachkommastelle.
Signifikante Fortschritte wurden im 14.
Jahrhundert durch Madhava von Sangamagrama erzielt, der Methoden entwickelte,
um \(\pi\) bis auf elf und später dreizehn Stellen genau zu berechnen.
In der chinesischen Mathematik wurde im 5.
Jahrhundert eine Genauigkeit von etwa sieben Dezimalstellen erreicht.
Im 15.
Jahrhundert erreichte Jamshīd al-Kāshī eine Genauigkeit von sechzehn Stellen.
Zu Beginn des 17.
Jahrhunderts konnte Ludolph van Ceulen \(\pi\) bis auf 35 Stellen berechnen,
und im 19.
Jahrhundert erreichte Jurij Vega sogar eine Genauigkeit von 126 Stellen.

Diese historischen Entwicklungen zeigen die Bedeutung und den anhaltenden
wissenschaftlichen Fortschritt in der Annäherung von \(\pi\).
Die genaue Berechnung von \(\pi\) bleibt nicht nur eine mathematische
Herausforderung, sondern ist auch für viele praktische Anwendungen relevant,
etwa in der Ingenieurwissenschaft, der Physik und der Informatik.

Ziel dieses Berichts ist es, verschiedene mathematische Annäherungsmethoden zur
Berechnung von \(\pi\) zu untersuchen und miteinander zu vergleichen.
Neben der bereits bekannten Leibnizreihe werden zwei weitere Verfahren
implementiert und analysiert.
Diese Verfahren werden hinsichtlich ihrer Laufzeit, ihres Speicherbedarfs,
ihres Konvergenzverhaltens und des Verhältnisses von Rechenaufwand zu
Approximationsgenauigkeit sowohl analytisch als auch experimentell verglichen.
\subsubsection{Annäherung}

\begin{definition}[Annäherungsfolge]

\end{definition}

\subsubsection{Warum Deckmal und nicht }

\subsubsection{Bessere und schlechtere Annäherungsfolgen}

\subsubsection{Sublunare Konvergence}

\section{Annäherungsalgorithmen}

\subsection{Monte-Carlo}

\subsection{Leibniz-Formel für \(\pi\)}

\subsection{Chudnovsky-Algorithmus}

\section{Experimente}

\subsection{Gesammelte Daten}

\subsection{Auswertung}

\section{Zusammenfassung}

\printbibliography

\end{document}
