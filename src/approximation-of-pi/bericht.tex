\documentclass{scrartcl}
\usepackage{scrhack}

\subject{Bericht}
\titlehead{%
  Humboldt-Universität zu Berlin\\
  Mathematisch-Naturwissenschaftliche Fakultät\\
  Institut für Mathematik
}
\title{Annäherungen von \(\pi\)}
\author{
  Eingereicht von M. van Straten und P. Merz
}
\date{\today}

\usepackage[ngerman]{babel}
\usepackage{amsmath}
\usepackage{amssymb}
\usepackage{amsthm}
\usepackage{microtype}
\usepackage{hyperref}
\usepackage{graphicx}
\usepackage{booktabs}
\usepackage[
backend=biber,
style=alphabetic,
sorting=ynt
]{biblatex}
\usepackage{pdfpages}

\addbibresource{./map.bib}

\newtheorem{definition}{Definition}

\begin{document}
\maketitle
\cleardoublepage{}
\tableofcontents
\cleardoublepage{}

\section{Einführung}

Die Konstante \(\pi\) hat eine natürliche Definition in der euklidischen
Geometrie als das Verhältnis zwischen dem Umfang und dem Durchmesser eines
Kreises \cite{wikipedia:Pi, wikipedia:Euclidean_geometry,
    wikipedia:Circumference, wikipedia:Diameter}.
Sie findet sich an vielen anderen Stellen in der Mathematik: zum Beispiel im
Gaußschen Integral, bei den komplexen Einheitswurzeln und in der
Cauchy-Verteilung in der Wahrscheinlichkeitstheorie
\cite{wikipedia:Gaussian_integral, wikipedia:Roots_of_unity,
    wikipedia:Cauchy_distribution, wikipedia:Probability}.

\subsection{Motivation}

Annäherungen für die mathematische Konstante \(\pi\) in der Geschichte der
Mathematik erreichten eine Genauigkeit von bis zu 0,04

Weitere Fortschritte wurden erst im 14.
Jahrhundert erzielt, als Madhava von Sangamagrama Annäherungen entwickelte, die
bis zu elf und dann dreizehn Stellen korrekt waren
\cite{wikipedia:Madhava_of_Sangamagrama}, bevor die Zeitrechnung begann
\cite{wikipedia:Approximation, wikipedia:Pi, wikipedia:History_of_mathematics}.
In der chinesischen Mathematik wurde dies im 5.
Jahrhundert auf Annäherungen verbessert, die etwa sieben Dezimalstellen
entsprachen \cite{wikipedia:Chinese_mathematics}.
Jamshīd al-Kāshī erreichte als Nächstes sechzehn Stellen
\cite{wikipedia:Jamshid_al-Kashi}.
Frühneuzeitliche Mathematiker erreichten eine Genauigkeit von 35 Stellen zu
Beginn des 17.
Jahrhunderts (Ludolph van Ceulen) \cite{wikipedia:Ludolph_van_Ceulen}, und 126
Stellen im 19.
Jahrhundert (Jurij Vega) \cite{wikipedia:Jurij_Vega}, was die für jede denkbare
Anwendung außerhalb der reinen Mathematik erforderliche Genauigkeit übertraf.
\\
Mit diesem Bericht möchten wir in die Fußstapfen unserer Mitmathematiker treten
und verschiedene theoretische Annäherungsmethoden zur Berechnung von \(\pi\)
untersuchen.

\subsection{Einführung in die Theorie}
\subsubsection{Annäherung}

\begin{definition}[Annäherungsfolge]

\end{definition}

\subsubsection{Warum Deckmal und nicht }

\subsubsection{Bessere und schlechtere Annäherungsfolgen}

\subsubsection{Sublunare Konvergence}

\section{Annäherungsalgorithmen}

\subsection{Monte-Carlo}

\subsection{Leibniz-Formel für \(\pi\)}

\subsection{Chudnovsky-Algorithmus}

\section{Experimente}

\subsection{Gesammelte Daten}

\subsection{Auswertung}

\section{Zusammenfassung}

\printbibliography

\end{document}
