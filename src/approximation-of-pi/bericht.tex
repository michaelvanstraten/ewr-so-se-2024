\documentclass{scrartcl}
\usepackage{scrhack}

\subject{Bericht}
\titlehead{%
  Humboldt-Universität zu Berlin\\
  Mathematisch-Naturwissenschaftliche Fakultät\\
  Institut für Mathematik
}
\title{Annäherungen von \(\pi\)}
\author{
  Eingereicht von M. van Straten und P. Merz
}
\date{\today}



\usepackage[ngerman]{babel}
\usepackage{amsmath}
\usepackage{amssymb}
\usepackage{amsthm}
\usepackage{float}
\usepackage{microtype}
\usepackage{hyperref}
\usepackage{graphicx}
\usepackage{booktabs}
\usepackage[
backend=biber,
style=alphabetic,
sorting=ynt
]{biblatex}
\usepackage{pdfpages}
\usepackage{svg}

\addbibresource{./bericht.bib}
\newtheorem{definition}{Definition}
\newtheorem{approximation sequence}{Annäherungsfolge}
\newtheorem{observation}{Beobachtung}

\parskip = \baselineskip
\parindent = 0pt

\begin{document}
\maketitle
\cleardoublepage{}
\tableofcontents
\cleardoublepage{}

\section{Motivation und Einleitung}

Die Konstante \(\pi\) hat eine natürliche Definition in der euklidischen
Geometrie als das Verhältnis zwischen dem Umfang und dem Durchmesser eines
Kreises.
Sie ist von fundamentaler Bedeutung in vielen Bereichen der Mathematik und
Physik.
Beispielsweise tritt \(\pi\) im Gaußschen Integral auf, bei den komplexen
Einheitswurzeln und in der Cauchy-Verteilung in der Wahrscheinlichkeitstheorie.
Die Relevanz von \(\pi\) erstreckt sich dabei über zahlreiche Disziplinen
hinweg, was sie zu einer der wichtigsten Konstanten in der Wissenschaft macht.

Historisch gesehen haben sich Mathematiker stets bemüht, \(\pi\) immer genauer
zu approximieren.
In der Antike reichte die Genauigkeit solcher Annäherungen oft nur bis zur
zweiten Nachkommastelle.
Signifikante Fortschritte wurden im 14.
Jahrhundert durch Madhava von Sangamagrama erzielt, der Methoden entwickelte,
um \(\pi\) bis auf elf und später dreizehn Stellen genau zu berechnen.
In der chinesischen Mathematik wurde im 5.
Jahrhundert eine Genauigkeit von etwa sieben Dezimalstellen erreicht.
Im 15.
Jahrhundert erreichte Jamshīd al-Kāshī eine Genauigkeit von sechzehn Stellen.
Zu Beginn des 17.
Jahrhunderts konnte Ludolph van Ceulen \(\pi\) bis auf 35 Stellen berechnen,
und im 19.
Jahrhundert erreichte Jurij Vega sogar eine Genauigkeit von 126 Stellen.

Diese historischen Entwicklungen zeigen die Bedeutung und den anhaltenden
wissenschaftlichen Fortschritt in der Annäherung von \(\pi\).
Die genaue Berechnung von \(\pi\) bleibt nicht nur eine mathematische
Herausforderung, sondern ist auch für viele praktische Anwendungen relevant,
etwa in der Ingenieurwissenschaft, der Physik und der Informatik.

Ziel dieses Berichts ist es, verschiedene mathematische Annäherungsmethoden zur
Berechnung von \(\pi\) zu untersuchen und miteinander zu vergleichen.
Neben der bereits bekannten Leibnizreihe werden zwei weitere Verfahren
implementiert und analysiert.
Diese Verfahren werden hinsichtlich ihrer Laufzeit, ihres Speicherbedarfs,
ihres Konvergenzverhaltens und des Verhältnisses von Rechenaufwand zu
Approximationsgenauigkeit sowohl analytisch als auch experimentell verglichen.

\section{Theorie}

\subsection{Einführung und Einordnung von Fachbegriffen}

\begin{definition}[Annäherungsfolge]
  Sei \((a_n)_{n \in \mathbb{N}}\) eine Folge reeller Zahlen und \(P \in \mathbb{R}\), sodass
  \[\lim_{n \to \infty} a_n = P \] gilt.
  Die Folge \(a_n\) heißt dann Annäherungsfolge für den Punkt \(P\)

\end{definition}

\begin{definition}[Uniforme Teilmenge]
    Eine uniforme Teilmenge \(U\) einer Menge \(M\) ist eine Teilmenge, in der
    jedes Element mit gleicher Wahrscheinlichkeit ausgewählt wird.
    Dies bedeutet, dass für jedes \(x \in M\) die Wahrscheinlichkeit \(P(x \in U)\)
    konstant ist.
\end{definition}

\begin{definition}[Bessere und schlechtere Annäherungsfolgen]
  Seien \((a_n)_{n \in \mathbb{N}}\) und \((b_n)_{n \in \mathbb{N}}\) zwei Annäherungsfolgen für den Punkt \(P \in \mathbb{R}\), d.h.
  \[\lim_{n \to \infty} a_n = P = \lim_{n \to \infty} b_n \] Die Annäherungsfolge \(a_n\) nennen wir besser bzw. schlechter als \(b_n\), falls ein
  \(N \in \mathbb{N}\) existiert, sodass für alle \(n \geq N\) gilt:
  \[ |(a_n - P)| < |(b_n - P)| \text{ bzw. } |(a_n - P)| > |(b_n - P)|\]
\end{definition}

\begin{definition}[Sublineare Konvergence]
\end{definition}

\subsection{Wahl eines angemessenen Datentypen}

Für die präzise Berechnung von \(\pi\) ist die Wahl eines angemessenen
Datentyps von entscheidender Bedeutung.
Datentypen wie \texttt{float32} oder \texttt{float64} bieten eine begrenzte
Genauigkeit, die für viele wissenschaftliche Anwendungen ausreichend sein mag,
aber für die exakte Annäherung von \(\pi\) über \(n\) beliebige Dezimalstellen
hinweg nicht ausreicht.
% TODO: Können wie hier die float Darstellung aus der Vorlesung nutzen?
Diese Datentypen nutzen eine feste Anzahl von Bits für die Mantisse und den
Exponenten, was ihre Präzision einschränkt.

\texttt{float32} hat eine Genauigkeit von etwa 7 Dezimalstellen, während \texttt{float64} etwa 16 Dezimalstellen bietet.
Um \(\pi\) jedoch auf eine exakte Anzahl von Stellen genau zu approximieren,
benötigt man eine höhere Präzision, die durch diese Datentypen nicht erreicht
werden kann.
Hier kommt der \texttt{Decimal}-Datentyp ins Spiel, der eine beliebig wählbare,
feste Mantissen Länge erlaubt.

Eine weitere benötigte Eigenschaft des \texttt{Decimal}-Datentyps ist die
Vermeidung von Rundungsfehlern, die bei den Standard-Gleitkomma-Datentypen
auftreten können Diese Fehler summieren sich bei iterativen Berechnungen, wie
sie zur Annäherung von \(\pi\) notwendig sind, und können zu signifikanten
Abweichungen führen.

\subsection{Annäherungsfolgen}

\begin{approximation sequence}[Monte-Carlo]

\end{approximation sequence}

\begin{approximation sequence}[Leibniz-Formel]

\end{approximation sequence}

\begin{approximation sequence}[Chudnovsky-Algorithmus]

\end{approximation sequence}

\section{Experimente}

\subsection{Konvergenzanalyse}

In diesem Experiment wird die Konvergenz verschiedener Pi-Annäherungsfolgen
analysiert. Ziel ist es, die Anzahl der korrekt approximierten Stellen von Pi
in Abhängigkeit von der Position innerhalb der Sequenz auf einer
logarithmischen Skala darzustellen. Die verwendeten Sequenzen umfassen Leibniz,
Monte Carlo, Gauss-Legendre und Chudnovsky.

Um die Ergebnisse zu reproduzieren, können die folgenden Parameter verwendet
werden:
\begin{verbatim}
approximation-of-pi convergence --precision 50 --stop 4
\end{verbatim}

\begin{figure}[H]
    \centering
    \includesvg[width=0.6\textwidth]{figures/convergence.svg}
    \caption{Konvergenzanalyse der verschiedenen Pi-Annäherungsmethoden in Abhängigkeit von der Position innerhalb der Sequenz (logarithmische Skala).}
    \label{fig:convergence-analysis}
\end{figure}

\begin{observation}
    Die Analyse zeigt, dass die Gauss-Legendre- und Chudnovsky-Sequenzen sehr
    schnell konvergieren und eine hohe Anzahl korrekt approximierter Stellen
    von Pi bereits bei niedrigen Positionen innerhalb der Sequenz erreichen.
    Die Gauss-Legendre-Sequenz zeigt hierbei die beste Konvergenz. Die
    Leibniz-Sequenz weist eine deutlich langsamere Konvergenz auf, während die
    Monte-Carlo-Sequenz erwartungsgemäß die geringste Anzahl korrekt
    approximierter Stellen liefert, was auf ihre probabilistische Natur
    zurückzuführen ist.
\end{observation}

\subsection{Speicherbedarf}

In diesem Experiment wird der Speicherbedarf der oben genannten
Annäherungsfolgen in Abhängigkeit von der Mantissenlänge analysiert. Jede
Sequenz wird hinsichtlich des Speicherbedarfs bei unterschiedlichen
Mantissenlängen untersucht.

Um die Ergebnisse zu reproduzieren, können die folgenden Parameter verwendet
werden:
\begin{verbatim}
python -m ewr_so_se_2024.approximation_of_pi memory-usage --digits 512
\end{verbatim}

\begin{figure}[H]
    \centering
    \includesvg[width=0.6\textwidth]{figures/memory-usage.svg}
    \caption{Speicherbedarf der verschiedenen Pi-Annäherungsfolgen in Abhängigkeit von der Anzahl der Dezimalstellen.}
    \label{fig:memory-usage}
\end{figure}

\begin{observation}
    Die Analyse zeigt, dass der Speicherbedarf mit zunehmender Anzahl der
    Dezimalstellen für alle Sequenzen linear ansteigt. Dabei ist der
    Speicherbedarf der Gauss-Legendre-Sequenz am höchsten, gefolgt von der
    Chudnovsky-Sequenz. Die Leibniz- und Monte-Carlo-Sequenzen zeigen einen
    vergleichsweise geringeren Anstieg des Speicherbedarfs. Besonders
    hervorzuheben ist, dass die Monte-Carlo-Sequenz den geringsten
    Speicherbedarf aufweist, was auf ihre probabilistische Natur und geringere
    Genauigkeit bei höherer Präzision hinweist.
\end{observation}

\section{Auswertung}

\section{Zusammenfassung}

\printbibliography

\end{document}
