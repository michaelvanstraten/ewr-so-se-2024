\documentclass{scrreprt}
\usepackage{caption}
\usepackage{subcaption}
\usepackage{scrhack}
\usepackage[utf8]{inputenc}
\KOMAoptions{
  % font size in pt
  fontsize=12,
  % parameter for size of type area, cf. KOMA script documentation
  DIV=13,
  % option for two-sided documents : false, semi, true
  twoside=false,
  % option for paragraph skip :
  %   false - indent, without vertical space between two paragraphs
  %   true  - vertical space between two paragraphs, without indent
  parskip=true,
  % print 'chapter' before the chapter number
  chapterprefix=true,
  % option to generate a draft version
  draft=false
}

\subject{Bericht Handout}
% head line of the title page, here with department and logo
\titlehead{%
  \begin{minipage}{.7\textwidth}%
  Humboldt-Universit\"at zu Berlin\\
  Mathematisch-Naturwissenschaftliche Fakult\"at\\
  Institut f\"ur Mathematik
  \end{minipage}
}
% title of the document
\title{Konvergenzanalyse der harmonischen Reihe\\
mithilfe von Maschinenzahlen}
% optional subtitle
%\subtitle{Draft from~\today}
% information about the author
\author{%
  Eingereicht von M. van Straten und P. Merz
}

\date{23.05.2024}


\usepackage[ngerman]{babel}


%%% MATH %%%%%%%%%%%%%%%%%%%%%%%%%%%%%%%%%%%%%%%%%%%%%%%%%%%%%%%%%%%%%%%%%%%%%%
% amsmath provides commands for type-setting mathematical formulas
\usepackage{amsmath}
% amssymb provides additional symbols
\usepackage{amssymb}
% HINT
% Use http://detexify.kirelabs.org/classify.html to find unknown symbols!


%%% MATH ENVIRONMENTS %%%%%%%%%%%%%%%%%%%%%%%%%%%%%%%%%%%%%%%%%%%%%%%%%%%%%%%%%
% amsthm provides environments for typical math text structure
\usepackage{amsthm}

%%% make tables with nice horizontal rules (avoid vertical rules)
\usepackage{booktabs}
%%% COLORS %%%%%%%%%%%%%%%%%%%%%%%%%%%%%%%%%%%%%%%%%%%%%%%%%%%%%%%%%%%%%%%%%%%%
% define own colors and use colored text
\usepackage[pdftex,svgnames,hyperref]{xcolor}


%%% FONT %%%%%%%%%%%%%%%%%%%%%%%%%%%%%%%%%%%%%%%%%%%%%%%%%%%%%%%%%%%%%%%%%%%%%%
% choose latin modern font (a good general purpose font)
\usepackage{lmodern}
% fontenc and microtype improve the appearance of the font
\usepackage[T1]{fontenc}
\usepackage{microtype}


%%% HYPERLINKS %%%%%%%%%%%%%%%%%%%%%%%%%%%%%%%%%%%%%%%%%%%%%%%%%%%%%%%%%%%%%%%%
% automatic generation of hyperlinks for references and URIs
\usepackage{hyperref}


%%% GRAPHICAL ELEMENTS %%%%%%%%%%%%%%%%%%%%%%%%%%%%%%%%%%%%%%%%%%%%%%%%%%%%%%%%
% provides commands to include graphics
\usepackage{graphicx}


%%% TABLES %%%%%%%%%%%%%%%%%%%%%%%%%%%%%%%%%%%%%%%%%%%%%%%%%%%%%%%%%%%%%%%%%%%%
% provides commands for good-looking tables
\usepackage{booktabs}

%%% Code Listings %%%%%%%%%%%%%%%%
% provides commands for including code (python, latex, ...)
\usepackage{listings}

\usepackage{tikz}

% Required for tikzplotlib
\usepackage{pgfplots}
\DeclareUnicodeCharacter{2212}{−}
\usepgfplotslibrary{groupplots,dateplot}
\usetikzlibrary{patterns,shapes.arrows}
\pgfplotsset{compat=newest}

\begin{document}
\maketitle
\tableofcontents
\cleardoublepage{}

\chapter{Einführung und Motivation}
Die harmonische Reihe, definiert als: \[ \sum_{n=1}^{\infty} \frac{1}{n} \] ist
eine klassische unendliche Reihe, die in vielen Bereichen der Mathematik und
Informatik von Interesse ist.
Obwohl sie divergiert, bietet die Untersuchung ihrer Partialsummen wertvolle
Einblicke in numerische Methoden und deren Präzision.
Das Konvergenzverhalten der Folge der Partialsummen bei Berechnung mittels
Computern ist von besonderem Interesse, da es die Grenzen und Genauigkeit
numerischer Berechnungen darstellt.
In dieser Arbeit untersuchen wir, wie verschiedene Algorithmen und Datentypen,
insbesondere diejenigen die von numpy bereitgestellt werden, die Berechnung der
Partialsummen der harmonischen Reihe beeinflussen.
Dies hat praktische Relevanz für numerische Mathematik, Algorithmendesign und
die Implementierung in wissenschaftlicher Software.
\chapter{Theoretische Grundlagen}

\section{Darstellung reeller Zahlen}
Computer stellen Zahlen mithilfe von Bits dar, die entweder den Wert 0 oder 1
annehmen können.
Die Anzahl der Bits hängt vom jeweiligen Rechner ab und ist heutzutage 64bit
bzw.
32bit.
Das heißt man kann nur eingeschränkt viele reelle Zahlen darstellen.
Diese durch einen Rechner dargestellten reellen Zahlen heißen Gleitkommazahlen.
Und die Menge aller Gleitkommazahlen \[\mathcal{M} = \mathbb{R}_N \cup
    \mathbb{R}_D \] bestehend aus der Vereinigung der normalisierten
Gleitkommazahlen und den subnormalen Gleitkommazahlen, also Zahlen, die sehr
nah an der Null liegen, heißt die Menge der Maschinenzahlen

\section{Runden von Gleitkommazahlen} Da die Anzahl an Bits, die für die
Darstellung einer Gleitkommazahl verantwortlich ist, beschränkt ist, können zum
einen einige Zahlen, gar nicht dargestellt werden, und zum anderen bedeutet
dies insbesondere, dass Maschinenzahlen unter den grundlegenden arithmetischen
Operation nicht abgeschlossen sind.
Daher muss eine Rundungsvorschrift \[rd_t: \mathbb{R}_N \cup \mathbb{R}_D
    \rightarrow \mathbb{R}_N \cup \mathbb{R}_D\] eingeführt werden, sodass man nach
diesen arithmetischen Operationen wieder eine Maschinenzahl erhält.
Für diese Rundungsvorschrift sind folgende Punkte wichtig:
\begin{itemize}
    \item Eine vorgegebene reelle Zahl wird zur nächstgelegen Maschinenzahl gerundet
    \item Falls eine Zahl genau zwischen zwei Maschinenzahlen liegt, so wird sie zur nächstgrößeren Maschinenzahl gerundet
\end{itemize}

\chapter{Experimente}
Definiere \[H_n := \sum_{i=1}^{n} \frac{1}{i} \] als die n-te harmonische
Summe.
Nach Ergenissen der Analysis, ist bekannt dass diese Folge an Partialsummen
bestimmt divergiert.
Deshalb haben wir uns angeschaut, wie das Ändern des Datentyps der Summanden,
sowie die Summationsmethode, das Konvergenzverhalten beeinflussen.
Hierfür haben wir diese Summe für mehrere Werte von n, die einen logarithmisch
konstanten Abstand haben, angeschaut.

\section{Verwendete Summationsmethoden}
\subsection{Vorwärtssummation}
Zur Berechnung der n-ten harmonischen Summe haben wir zum einen die
Vorwärtssummation benutzt.
Dies ist die am einfachsten, jedoch aber auch die naivste Methode, Summen
auszurechnen.
Die Vorwärtssummation, erfolgt durch Addieren der Summanden, die jeweils in
einer Klammer stehen.
\[ H_n = (...((\frac{1}{1} + \frac{1}{2}) + \frac{1}{3}) + ... ) + \frac{1}{n} \]
\subsection{Kahansummation}
Der Kahansummation Algorithmus addiert zwar auch wie die Vorwärtssummation die
einzelnen Summanden nacheinander, doch wird durch Einführung einer
Korrektur-Variable dem Fehler, der durch die Rundungsvorschrift entsteht,
entgegengewirkt.
\section{Daten}

\subsubsection{Summation für Datentyp np.float32}
\input{forward_float32.tex}
\input{kahan_float32.tex}

\subsubsection{Summation für Datentyp np.float64}
\input{forward_float64.tex}
\input{kahan_float64.tex}

\section{Auswertung}
Wie zu beobachten ist, stagniert die harmonische Summe bei der
Vorwärtssummatiion für den Datentyp np.float32 und bleibt irgendwann konstant,
während sie für die Kahansummation noch weiter wächst.
Für den Datentyp np.float64 hingegen wachsen beide Summationsmethoden und
stagnieren nicht für unsere Eingaben von n.

Eine Erklärung dafür ist, dass die einzelnen Terme der harmonischen Reihe
irgendwann so klein sind, dass die Präzision vom Datentyp np.float32, nicht
ausreicht und die Summanden kleiner als die kleinste von Null verschiedene
Maschinenzahl des Typs np.float32 sind, und dann auf die Null abgerundet
werden.
Das heißt ab einem bestimmten Punkt werden bei der Vorwärtssummation nur noch
Nullen summiert, was nichts an dem Wert der Reihe ändert.

Bei der Kahan Summation hingegen stagniert die Summe jedoch trotz Datentyps
np.float32 nicht, was diesem Korrekturterm zuzuschreiben ist, da dieser dem
Runden und den dadurch entstehenden Fehlern entgegengewirkt.

Für den Datentyp np.float64 ist die Präzision deutlich höher, und die einzelnen
Summanden werden daher auch für sehr große n nicht auf die Null abgerundet, was
sich dann in einem stetigen Wachstum der harmonischen Reihe wiederspiegelt.

\chapter{Zusammenfassung}
Zusammengefasst haben wir in dieser Untersuchung zur harmonischen Reihe
gesehen, dass der jeweilige Datentyp mit dem man seine Rechnungen durchführt
eine bedeutende Rolle spielt.
Ferner haben wir jedoch auch erkennnen können, dass die Wahl der
Summationsmethode wichtig ist, und eben auch dafür sorgen kann, dass man
weitestgehend Präzision behält, obwohl man einen eher unpräzisen Datentyp
verwendet.
Von weiterem Interesse wäre es diese Analyse auf eine Vielzahl weiterer
Datentypen und noch bessere Summationsalgorithmen auszuweiten.

\chapter{Literaturverzeichnis}
\url{https://en.wikipedia.org/wiki/Kahan_summation_algorithm} [letzer Aufruf: 20.05.2024]
\end{document}

