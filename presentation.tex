\documentclass{beamer}
\usepackage[ngerman]{babel} 
\usepackage{textgreek}
\usepackage{amsmath}
\usepackage{amssymb}
\usepackage[
backend=biber,
style=alphabetic,
sorting=ynt
]{biblatex}
\bibliography{presentation.bib}

% Disable navigation buttons
\setbeamertemplate{navigation symbols}{}

\title{Approximation von \textpi}
\author{M. van Straten und P. Merz}
\institute{Humboldt-Universität zu Berlin \\
           Sommersemester 2024}
\date{\today}

\usetheme{Berlin}


\begin{document}

\maketitle

\begin{frame}{Inhalt}
    \begin{itemize}
        \item<1-> Einführung
        \item<2-> Theorie
        \item<3-> Annäherungsalgorithmen
        \item<4-> Experimente
        \item<5-> Daten
        \item<6-> Auswertung
    \end{itemize}
\end{frame}

\begin{frame}{Einführung}
Die Kreiszahl \(\pi\) findet in vielen Bereichen der Mathematik Anwendung:
    \begin{itemize}
        \item<1-> Geometrie
        \item<2-> Wahrscheinlichkeitstheorie
        \item<3-> Cauchy-Verteilung \cite{Cauchy}
\end{itemize}
Aber: Zum Berechnen konkreter Werte muss \(\pi\) gerundet werden
\end{frame}

\begin{frame}{Einführung: Historie}
    \begin{tabular}{|c||c||c|}
        \hline
            Datum & Person & Korrekte Dezimalstellen \\ 
        \hline
            ca. 250 v. Chr. & Archimedes & 2\\ 
        \hline
            1400 & Madhava & 10\\
        \hline
            1981 & Jean Guilloud & 2,000,050\\ 
        \hline
            Januar 1988 & \shortstack{Yasumasa Kanada und \\ Yoshiaki Tamura} & 201.326.551\\
        \hline
            31.12.2009 & Fabrice Bellard & 2,699,999,990,000\\
        \hline
            14.03.2024 & \shortstack{Jordan Ranous, \\
             Kevin O’Brien \\ und Brian Beeler} & 105,000,000,000,000\\                                                              
        \hline      
    \end{tabular}
    \cite{Chronology}
\end{frame}

\begin{frame}{Einführung: Motivation}
Wie viele Nachkommastellen von \(\pi\) benötigt man?
    \begin{itemize}
        \item<2-> NASA JPL benutzt 15 Nachkommastellen \cite{NASA}                                                                                         
    \end{itemize}
\uncover<3->{Warum also \(\pi\) genauer approximieren?}
 \begin{itemize}
    \item<4-> Effizientere Approximationsalgorithmen wurden mit der Zeit entdeckt
    \item<5-> Untersuchung der Effizienz, sowie Vergleich der Genaugikeit mit dem Rechenaufwand dieser Algorithmen 
              um Grenzen der modernen Rechner zu testen
\end{itemize}
\end{frame}



\begin{frame}{Theorie: Definitionen}
    \begin{block}{Definition von \(\pi\) in der euklidischen Geometrie}
        Verhältnis zwischen Umfang und Durchmesser eines Kreises
    \end{block}
    Äquivalente Definitionen z. B. in der Analysis über bestimmte Nullstellen trigonometrischer Funktionen                                            %(siehe Analysis I Skript)
    \\[10pt]
\uncover<2-> {
    \begin{block}{Definition eines Algorithmus}
        Handlungsvorschrift bestehend aus einer Menge an wohldefinierten Schritten                                                                                             %https://de.wikipedia.org/wiki/Algorithmus see citation [1]
    \end{block}
     Nützlich, da Algorithmen von Rechnern ausgeführt werden können}
\end{frame}


\begin{frame}{Theorie: Annäherungsalgorithmen}
    \begin{itemize}
        \item Monte-Carlo Simulation
        \item Leibniz-Reihe
        \item Gauß-Legendre
    \end{itemize}
\end{frame}

\begin{frame}{Annäherungsalgorithmen: Monte-Carlo Simulation}
    Basierend auf dem Gesetz der Großen Zahl\footnote{Das Gesetz der Großen Zahl
        besagt, dass sich die relative Häufigkeit eines Ereignisses bei großer Anzahl
        von Versuchen dem Erwartungswert annähert.
    } wird versucht, über eine unbekannte Menge \(M\) analytische Aussagen zu treffen,
    indem man sich auf eine endliche uniforme Teilmenge beschränkt.
    \begin{itemize}
        \item<1-> In diesem Fall ist \(M \coloneq \{(x, y) \mid 0 \le x,y \le 1\}\)
        \item<2-> Mit dem Flächeninhalt eines Kreises folgt, dass
              \begin{equation*}
                  \frac{\pi}{4} = \frac{
                      |\{(x,y) \in M \only<3->{\text{\ mit } \sqrt{x^2 + y^2} \le 1} \}|
                  }{|M|}
              \end{equation*}
              ist.
        \item<4-> \(\pi\) lässt sich somit approximieren, indem dieses Verhältnis über eine große uniforme Teilmenge von \(M\) gebildet wird.
    \end{itemize}
\end{frame}


\begin{frame}{Annäherungsalgorithmen: Leibniz-Reihe}
    Durch Ergebnisse der Analysis leitete Leibniz folgedendes Ergebnis her \cite{Leibniz}                                                                                                       
    \[ \pi = \sum_{k=0}^{\infty} \frac{(-1)^k}{2k+1} \] 
    \uncover<2->{Annäherung von \(\pi\) über \(n\)-te Partialsumme \(\pi \approx \sum_{k=0}^{n} \frac{(-1)^k}{2k+1}\) }                                                                                              
    \begin{itemize}
        \item<3-> Konvergiert sehr langsam, präzise: Sublineare Konvergenz \\
        \item<3-> Benötigt grob 5 Milliarden Terme um auf 10 korrekte Nachkommastellen zu approximieren
    \end{itemize}
\end{frame}



\begin{frame}{Annäherungsalgorithmen: Gauß-Legendre}                                                                                                        %https://web.archive.org/web/20080726033059/http://wwwmaths.anu.edu.au/~brent/pd/rpb028.pdf
    \begin{itemize}                                                                                                                                         
        \item<1-> Approximiert \(\pi\) mittels Folgen, die sich des arithmetischen- und geometrischen Mittels zweier Zahlen bedienen
        \item<2-> \( a_0 = 1 \;\;\; b_0 = \frac{1}{\sqrt{2}} \;\;\; t_0 = \frac{1}{4} \;\;\; p_0 = 1 \)
        \item<3-> \( a_{n+1} = \frac{a_n + b_n}{2} \)
        \item<4-> \( b_{n+1} = \sqrt{a_nb_n} \)
        \item<5-> \( t_{n+1} = t_n - p_n(a_n - a_{n+1})^2 \)
        \item<6-> \( p_{n+1} = 2p_n \)
        \item<7-> \(\pi\) wird dann approximiert durch \[ \pi \approx \frac{(a_{n+1} + b_{n+1})^2}{4t_{n+1}} \]
        \item<8-> Konvergiert quadratisch gegen \(\pi\) \cite{Gauß-Legendre}
    \end{itemize}   
\end{frame}


\begin{frame}{Experimente}  
    Setting: Datentyp Decimal
\end{frame}

\begin{frame}{Daten}

\end{frame}



\begin{frame}{Auswertung und Zusammenfassung}

\end{frame}

\begin{frame}{Literaturverzeichnis}
\printbibliography
\end{frame}


\end{document}
