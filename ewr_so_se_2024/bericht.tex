\documentclass{scrartcl}
\usepackage{scrhack} 
\usepackage[utf8]{inputenc}

\subject{Bericht}
\titlehead{%
  \begin{minipage}{.7\textwidth}%
  Humboldt-Universit\"at zu Berlin\\
  Mathematisch-Naturwissenschaftliche Fakult\"at\\
  Institut f\"ur Mathematik
  \end{minipage}
}
\title{Konvergenzanalyse der harmonischen Reihe\\
mithilfe von Maschinenzahlen}
\author{
  Eingereicht von M. van Straten und P. Merz
}
\date{06.06.2024}

\usepackage[ngerman]{babel}
\usepackage{amsmath}
\usepackage{amssymb}

\usepackage{amsthm}

% choose latin modern font (a good general purpose font)
\usepackage{lmodern}
% fontenc and microtype improve the appearance of the font
\usepackage[T1]{fontenc}
\usepackage{microtype}
% automatic generation of hyperlinks for references and URIs
\usepackage{hyperref}
% provides commands to include graphics
\usepackage{graphicx}
% provides commands for good-looking tables
\usepackage{booktabs}

\begin{document}

% generating the title page
\maketitle
% generating the table of contents (requires to run pdflatex twice!)
\cleardoublepage{}
\tableofcontents
% start a new page
\cleardoublepage{}

\section{Einleitung und Motivation}
Die harmonische Reihe, definiert als: \[ \sum_{n=1}^{\infty} \frac{1}{n} \] ist
eine klassische unendliche Reihe, die in vielen Bereichen der Mathematik und
Informatik von Interesse ist.
Obwohl sie divergiert, bietet die Untersuchung ihrer Partialsummen wertvolle
Einblicke in numerische Methoden und deren Präzision.
Das Konvergenz verhalten der Folge der Partialsummen bei Berechnung mittels
Computern ist von besonderem Interesse, da es die Grenzen und Genauigkeit
numerischer Berechnungen darstellt.
In dieser Arbeit untersuchen wir, wie verschiedene Algorithmen und Datentypen,
insbesondere diejenigen die von NumPy bereitgestellt werden, die Berechnung der
Partialsummen der harmonischen Reihe beeinflussen.
Dies hat praktische Relevanz für numerische Mathematik, Algorithmendesign und
die Implementierung in wissenschaftlicher Software.
% HIER CITE BERICHT HANDOUT

\section{Theorie}
\subsection{Darstellung reeller Zahlen auf Rechnern}
Wenn man an reelle Zahlen denkt, dann meistens in Basis \(10\), das heißt
reelle Zahlen werden mit den Ziffern aus \(B_{10} = \{0, 1, 2, \dots, 9\} \)
beschrieben.
Computer stellen reelle Zahlen hingegen mit Bits dar, die entweder \(0\) oder
\(1\) annehmen können.
Die Anzahl der Bits hängt vom jeweiligen Rechner bzw.
dem gewählten Datentypen
ab, typisch sind aber 32Bit bzw.\ 64Bit.
Da man aber nur beschränkt viele Bits hat, folgt, dass man nur beschränkt viele
reelle Zahlen darstellen kann.
Wir unterscheiden zwischen den normalisierten Gleitkommazahlen, und den
subnormalen Gleitkommazahlen.
\begin{flushleft}
    Für eine normalisierte \(t\)-stellige Gleitkommazahl \(x\) gilt:
\end{flushleft} \[ x = (-1)^{v}\sum_{i=1}^{t} x_i 2^{-i} = (-1)^v 2^N (0.x_1x_2
    .
    .. x_t)_2 \]

\begin{flushleft}
    für \(v \in \{0, 1\}, N \in \mathbb{Z}, -1021 < N \leqslant 1024\) und \(\forall i = 1,...,n: x_i \in \{0, 1\} \). \\
    Die Zahl \( m =  \sum_{i=1}^{t}x_i 2^{-i} = (0.x_1x_2...x_t)_2 \) wird die Mantisse von \(x\) und \(t\) die Mantissenlänge genannt.        \\
    Man schreibt \(x \in \mathbb{R}_N \), die Menge der normalisierten Gleitkommazahlen.
    Für eine subnormale Gleitkommazahl \(x\) gilt: \end{flushleft} \[x = (-1)^v
    2^{N_{min}} (0.x_2x_3.
    % CITE MOODLE KURS
    ..x_t) = (-1)^v 2^{N_{min}+1} \sum_{i=1}^{t} 2^{-i}x_i \]
mit \(x_1 = 0\) und \(N_{min} + 1 = -1021 \)
\begin{flushleft}
    Man schreibt \(x \in \mathbb{R}_N \), die Menge der subnormalen
    Gleitkommazahlen.
    % CITE MOODLE KURS

    Man definiert die Menge der Maschinenzahlen als \[\mathcal{M} = \mathbb{R}_N
        \cup \mathbb{R}_D \] also der Vereinigung der normalisierten Gleitkommazahlen
    und der subnormalen Gleitkommazahlen.
    % CITE MOODLE KURS
\end{flushleft}

\subsubsection{np.float16, np.float32 und np.float64}
Bei den verschiedenen Float Typen handelt es sich von Bibliothek Rugby
bereitgestellte Dateitypen.
Diese Datentypen stellen auch Gleitkommazahlen dar, und wie die Namen auch
verraten entweder mit \(16\), \(32\) oder \(64\)Bit.
Die Bits werden für die verschiedenen Typen wie folgt verteilt:
Für float16
\begin{itemize}
    \item 1 Bit für das Vorzeichen der Zahl                                                               %CITE https://en.wikipedia.org/wiki/Half-precision_floating-point_format
    \item 5 Bits für den Exponenten
    \item 10 Bits für die Mantisse
\end{itemize}
Für float32
\begin{itemize}
    \item 1 Bit für das Vorzeichen der Zahl                                                                 % CITE MOODLE KURS FÜR float32 UND float64
    \item 8 Bits für den Exponenten
    \item 23 Bits für die Mantisse
\end{itemize}
und für float64
\begin{itemize}
    \item 1 Bit für das Vorzeichen der Zahl
    \item 11 Bits für den Exponenten
    \item 52 Bits für die Mantisse
\end{itemize}
Es ist also sehr einfach zu sehen, dass Zahlen des Datentyps float64 mit
höherer Präzision dargestellt werden können, als die der anderen Datentypen.

\subsection{Runden von Gleitkommazahlen und Rundungsfehler}
Da nur beschränkt viele Bits zur Darstellung von Gleitkommazahlen verwendet
werden, können einige reelle Zahlen gar nicht dargestellt werden.
Zum Beispiel ist \\ \((0.1)_{10} = (0.
\overline{0011})_2 \).
Da sich in der Darstellung die \(0011\) wiederholt, kann diese Zahl nicht exakt
vom Rechner dargestellt werden; sie muss also gerundet werden.
Dafür wird eine Rundungsvorschrift \[rd_t: \mathbb{R}_N \cup \mathbb{R}_D
    \rightarrow \mathbb{R}_N \cup \mathbb{R}_D\] mit \(t\) Mantissenlänge,
eingeführt, damit die Verknüpfung zweier Maschinenzahlen auch wieder eine
Maschinenzahl ist.
\\
Diese Rundungsvorschrift erfüllt folgende Eigenschaften
\begin{itemize}
    \item Eine vorgegebene reelle Zahl wird zur nächstgelegen Maschinenzahl gerundet
    \item Falls eine Zahl genau zwischen zwei Maschinenzahlen liegt, so wird sie zur nächstgrößeren Maschinenzahl gerundet.
\end{itemize}
Außerdem kann man herleiten, dass \[rd_t(x) = x(1+\varepsilon(x)) =
    \frac{x}{1-\eta(x)}\] mit \[\varepsilon(x) = \frac{rd_t(x) - x}{x} \text{ und}\
    \eta(x) = \frac{rd_t(x) - x}{rd_t(x)} \]

\subsection{Summationsmethoden} \subsubsection{Vorwärtssummation} Die
Vorwärtssummation ist, wie der Name auch sagt, eine recht simple Methode Zahlen
aufzuaddieren.
Die Terme werden hintereinander addiert, das heißt zuerst werden die ersten
beiden Summanden addiert, der Rechner entscheidet dann, ob es sich um eine
Maschinenzahl handelt, und falls nicht, rundet er auf die nächste
Maschinenzahl.
Danach erst wird die dritte Zahl auf diese, gegebenenfalls neue Zahl,
aufaddiert und es wird wieder geschaut, ob es sich um eine Maschinenzahl
handelt.
Dieser Prozess wird bis zum letzten Summanden durchgeführt.
Mithilfe von Klammerung lässt sich dies für die harmonische Reihe wie folgt
ausdrücken.
\[ \sum_{k=1}^{n} \frac{1}{k} = ((...((1+\frac{1}{2})+\frac{1}{3})+ ...) + \frac{1}{n}) \]
\subsubsection{Kahansummation}
Da die Maschinenzahlen unter den vier grundlegenden arithmetischen Operationen
nicht abgeschlossen, und der Rechner möglicherweise Runden muss, so wird sich,
vor allem bei einer sehr großen Anzahl an Zahlen, bei der Addition ein
Genauigkeitsfehler einschleichen.
Um diesen Fehler auszugleichen wird bei der Kahan Summation eine Laufvariable
eingeführt, in die nach jeder Addition der Fehler eingespeichert wird, um ihn
in der nächsten Addition wieder zu beheben.
Natürlich kann diese Laufvariable nur so genau sein, wie der jeweilige Datentyp
mit dem gerechnet wird.

\section{Experimente}
\subsection{Versuchsaufbau}
Um zu untersuchen, wie sich das Verhalten von Gleitkommazahlen auf die
Konvergenz der harmonischen Reihe auswirken, haben wir mittels Python zwei
Programme zu den eben vorgestellten Summationsmethoden geschrieben.
Aufsummiert haben wir bis zu verschiedenen Werten, die logarithmisch den
gleichen Abstand haben.
Generiert haben wir uns diese Zahlen mittels unseres pylogspace Programms,
welches eine Liste an ganzen Zahlen in einer gegebenen Basis (hier
standardmäßig \(10)\) zwischen einer oberen und unteren Grenze generiert.

\subsection{Daten}

Summation für Datentyp np.float16

Summation für Datentyp
np.float32

Summation für Datentyp np.float64

\subsection{Beobachtungen}

\section{Auswertung}

\section{Zusammenfassung}
Zusammengefasst kann man sagen, dass der Rechner reelle Zahlen im Binärsystem
darstellt und diese je nach Größe entweder als normalisierte oder subnormale
Gleitkommazahl darstellt.
Abhängig von dem gewählten Datentypen können Zahlen genauer dargestellt werden,
da ihnen mehr Bits zur Verfügung gestellt.
Außerdem sind diese Maschinenzahlen nicht abgeschlossen bezüglich Addition,
Subtraktion, Multiplikation oder Division, weshalb die Summe, das Produkt,
etc.\ vom Rechner auf die nächstgelegene Gleitkommazahl gerundet werden muss.
\begin{flushleft}
    Die Auswirkungen auf die harmonische Summe, die bekannterweise divergiert, sind
    sehr deutlich.
    Die Summanden werden sehr klein und werden für bestimmte Datentypen dann auf
    die Null abgerundet, was dazu führt, dass die harmonische Reihe für eben diese
    Datentypen konvergiert, da ab einem bestimmten Punkt nur noch Nullen addiert
    werden.
    \\
    Die Genauigkeit das Aufsummieren kann, im Vergleich zur vorwärts Summation, erhöht werden, indem man eine andere
    Summationsmethode verwendet, wie zum Beispiel die Kahan Summation.
    Dadurch kann das Konvergieren zwar verzögert, aber nicht gestoppt werden, da
    die Kahan Summation von einer Laufvariable abhängt, die den Rundungsfehler
    ausgleicht, aber die jedoch nur so genau sein kann wie der jeweilige Datentyp.
\end{flushleft}

\end{document}
