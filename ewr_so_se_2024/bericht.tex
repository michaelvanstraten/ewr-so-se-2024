\documentclass{scrartcl}
\usepackage{scrhack}
\usepackage[utf8]{inputenc}

\subject{Bericht}
\titlehead{%
  \begin{minipage}{.7\textwidth}%
  Humboldt-Universit\"at zu Berlin\\
  Mathematisch-Naturwissenschaftliche Fakult\"at\\
  Institut f\"ur Mathematik
  \end{minipage}
}
\title{Konvergenzanalyse der harmonischen Reihe\\
mithilfe von Maschinenzahlen}
\author{
  Eingereicht von M. van Straten und P. Merz
}
\date{06.06.2024}

\usepackage[ngerman]{babel}
\usepackage{amsmath}
\usepackage{amssymb}
\usepackage{amsthm}
\usepackage{lmodern}
\usepackage[T1]{fontenc}
\usepackage{microtype}
\usepackage{hyperref}
\usepackage{graphicx}
\usepackage{booktabs}
\usepackage[
backend=biber,
style=alphabetic,
sorting=ynt
]{biblatex}
\usepackage{pdfpages}

\addbibresource{./ewr_so_se_2024/bericht.bib}

\begin{document}

\maketitle
\cleardoublepage{}
\tableofcontents
\cleardoublepage{}

\section{Einleitung und Motivation}
Die harmonische Reihe, definiert als: \[ \sum_{n=1}^{\infty} \frac{1}{n} \] ist
eine klassische unendliche Reihe, die in vielen Bereichen der Mathematik und
Informatik von Interesse ist.
Obwohl sie divergiert, bietet die Untersuchung ihrer Partialsummen wertvolle
Einblicke in numerische Methoden und deren Präzision.
Das Konvergenzverhalten der Folge der Partialsummen bei Berechnung mittels
Computern ist von besonderem Interesse, da es die Grenzen und Genauigkeit
numerischer Berechnungen darstellt.
In dieser Arbeit untersuchen wir, wie verschiedene Algorithmen und Datentypen,
insbesondere diejenigen, die von NumPy bereitgestellt werden, die Berechnung
der Partialsummen der harmonischen Reihe beeinflussen.
Dies hat praktische Relevanz für numerische Mathematik, Algorithmendesign und
die Implementierung in wissenschaftlicher Software.

\section{Theorie}
\subsection{Darstellung reeller Zahlen auf Rechnern}
Reelle Zahlen werden üblicherweise im Dezimalsystem (Basis 10) dargestellt, das
heißt, sie werden mit den Ziffern aus \( B_{10} = \{0, 1, 2, \dots, 9\} \)
beschrieben.
Computer hingegen stellen reelle Zahlen mit Bits dar, die entweder \(0\) oder
\(1\) annehmen können.
Die Anzahl der Bits hängt vom jeweiligen Rechner bzw.
\ dem gewählten Datentyp ab, typisch sind aber 32 Bit bzw. 64 Bit.
Da man aber nur beschränkt viele Bits hat, folgt, dass man nur beschränkt viele
reelle Zahlen darstellen kann.
Wir unterscheiden zwischen den normalisierten Gleitkommazahlen und den
subnormalen Gleitkommazahlen.

Für \( v \in \{0, 1\} \) und \( N \in \mathbb{Z} \) mit \( -1021 < N \leq 1024 \) gilt für eine normalisierte \( t \)-stellige Gleitkommazahl \( x \):
\begin{align*}
    x & = {(-1)}^v \sum_{i=1}^t x_i 2^{-i}                                                                                                \\
      & = (-1)^v 2^N (0.x_1 x_2 \ldots x_t)_2 \qquad \text{für alle} \qquad i = \{1, \ldots, t\} \quad \text{mit} \quad x_i \in \{0, 1\}.
\end{align*}

Die Zahl \( m = \sum_{i=1}^{t}x_i 2^{-i} = (0.x_1x_2\ldots x_t)_2 \) wird die
Mantisse von \( x \) und \( t \) die Mantissenlänge genannt.
Man schreibt \( x \in \mathbb{R}_N \), die Menge der normalisierten
Gleitkommazahlen.
Für eine subnormale Gleitkommazahl \( x \) gilt: \[ x = (-1)^v 2^{N_{min}}
    (0.x_2 x_3 \ldots x_t) = (-1)^v 2^{N_{min}+1} \sum_{i=1}^{t} 2^{-i} x_i \] mit
\( x_1 = 0 \) und \( N_{min} + 1 = -1021 \).
Man schreibt \( x \in \mathbb{R}_D \), die Menge der subnormalen
Gleitkommazahlen.

Die Menge der Maschinenzahlen wird definiert als: \[ \mathcal{M} = \mathbb{R}_N
    \cup \mathbb{R}_D \] also der Vereinigung der normalisierten und der
subnormalen Gleitkommazahlen.

\subsubsection{np.float16, np.float32 und np.float64}
Bei den verschiedenen Float-Typen handelt es sich um von der Bibliothek NumPy
bereitgestellte Datentypen.
Diese Datentypen stellen auch Gleitkommazahlen dar, und wie die Namen schon
verraten, entweder mit \( 16 \), \( 32 \) oder \( 64 \) Bit.
Die Bits werden für die verschiedenen Typen wie folgt verteilt:

Für float16 \cite{half-precision_floating-point_format}:
\begin{itemize}
    \item 1 Bit für das Vorzeichen der Zahl % CITE https://en.wikipedia.org/wiki/Half-precision_floating-point_format
    \item 5 Bits für den Exponenten
    \item 10 Bits für die Mantisse
\end{itemize}

Für float32 \cite{single-precision_floating-point_format}:
\begin{itemize}
    \item 1 Bit für das Vorzeichen der Zahl % CITE MOODLE KURS FÜR float32 UND float64
    \item 8 Bits für den Exponenten
    \item 23 Bits für die Mantisse
\end{itemize}

Für float64 \cite{double-precision_floating-point_format}:
\begin{itemize}
    \item 1 Bit für das Vorzeichen der Zahl
    \item 11 Bits für den Exponenten
    \item 52 Bits für die Mantisse
\end{itemize}

Es ist also sehr einfach zu sehen, dass Zahlen des Datentyps float64 mit
höherer Präzision dargestellt werden können als die der anderen Datentypen.

\subsection{Runden von Gleitkommazahlen und Rundungsfehler}
Da nur beschränkt viele Bits zur Darstellung von Gleitkommazahlen verwendet
werden, können einige reelle Zahlen gar nicht dargestellt werden.
Zum Beispiel ist \( (0.1)_{10} = (0.
\overline{0011})_2 \).
Da sich in der Darstellung die \(0011\) wiederholt, kann diese Zahl nicht exakt
vom Rechner dargestellt werden; sie muss also gerundet werden.
Dafür wird eine Rundungsvorschrift
\begin{equation*}
    rd_t: \mathbb{R}_N \cup \mathbb{R}_D \rightarrow \mathbb{R}_N \cup \mathbb{R}_D
\end{equation*}
mit \(t\) Mantissenlänge eingeführt, damit die Verknüpfung zweier Maschinenzahlen auch wieder eine
Maschinenzahl ist.

Diese Rundungsvorschrift erfüllt folgende Eigenschaften:
\begin{itemize}
    \item Eine vorgegebene reelle Zahl wird zur nächstgelegenen Maschinenzahl gerundet.
    \item Falls eine Zahl genau zwischen zwei Maschinenzahlen liegt, so wird sie zur nächstgrößeren Maschinenzahl gerundet.
\end{itemize}

Außerdem kann man herleiten, dass: \[ rd_t(x) = x(1+\varepsilon(x)) =
    \frac{x}{1-\eta(x)} \] mit \[ \varepsilon(x) = \frac{rd_t(x) - x}{x} \text{ und
    } \eta(x) = \frac{rd_t(x) - x}{rd_t(x)} \]

\subsection{Summationsmethoden} \subsubsection{Vorwärtssummation} Die
Vorwärtssummation ist, wie der Name auch sagt, eine recht simple Methode,
Zahlen aufzuaddieren.
Die Terme werden hintereinander addiert, das heißt zuerst werden die ersten
beiden Summanden addiert, der Rechner entscheidet dann, ob es sich um eine
Maschinenzahl handelt, und falls nicht, rundet er auf die nächste
Maschinenzahl.
Danach erst wird die dritte Zahl auf diese, gegebenenfalls neue Zahl,
aufaddiert und es wird wieder geschaut, ob es sich um eine Maschinenzahl
handelt.
Dieser Prozess wird bis zum letzten Summanden durchgeführt.
Mithilfe von Klammerung lässt sich dies für die harmonische Reihe wie folgt
ausdrücken:
\begin{equation*}
    \sum_{k=1}^{n} \frac{1}{k} = ((\ldots((1 + \frac{1}{2}) + \frac{1}{3}) + \ldots) + \frac{1}{n}).
\end{equation*}

\subsubsection{Kahansummation}
Da die Maschinenzahlen unter den vier grundlegenden arithmetischen Operationen
nicht abgeschlossen sind und der Rechner möglicherweise runden muss, so wird
sich, vor allem bei einer sehr großen Anzahl an Zahlen, bei der Addition ein
Genauigkeitsfehler einschleichen.
Um diesen Fehler auszugleichen, wird bei der Kahan-Summation eine Laufvariable
eingeführt, in die nach jeder Addition der Fehler eingespeichert wird, um ihn
in der nächsten Addition wieder zu beheben.
Natürlich kann diese Laufvariable nur so genau sein, wie der jeweilige
Datentyp, mit dem gerechnet wird.

\section{Experimente}

\subsection{Versuchsaufbau}
Um zu untersuchen, wie sich das Verhalten von Gleitkommazahlen auf die
Konvergenz der harmonischen Reihe auswirken, haben wir mittels Python zwei
Programme zu den eben vorgestellten Summationsmethoden geschrieben.
Aufsummiert haben wir bis zu verschiedenen Werten, die logarithmisch den
gleichen Abstand haben.
Generiert haben wir uns diese Zahlen mittels unseres pylogspace-Programms,
welches eine Liste an ganzen Zahlen in einer gegebenen Basis (hier
standardmäßig \(10\)) zwischen einer oberen und unteren Grenze generiert.

\subsection{Daten}

\begin{center}
    \makebox[\textwidth]{\includegraphics[width=\textwidth]{./output.png}}
\end{center}

\subsection{Beobachtungen zu den Summationsmethoden und Datentypen}

Die folgenden Beobachtungen basieren auf den Ergebnissen der Summationsmethoden
(Vorwärtssummation und Kahan-Summation) für die Datentypen \texttt{float16} und
\texttt{float32}.

\subsubsection{Forward Summation using float16}

\textbf{Verhalten}
\begin{itemize}
    \item Die Summation konvergiert schnell und erreicht einen Punkt, ab dem keine signifikante Änderung mehr erfolgt.
    \item Die Konvergenz tritt bereits bei relativ kleinen Summanden (ab etwa \(10^3\)) ein.
\end{itemize}

\textbf{Ursache}
\begin{itemize}
    \item Bei der Verwendung von \texttt{float16} treten aufgrund der geringen Präzision früh Rundungsfehler auf.
          Diese führen dazu, dass kleine Summanden ab einem bestimmten Punkt ignoriert
          werden, da sie auf Null gerundet werden.
\end{itemize}

\textbf{Ergebnis}
\begin{itemize}
    \item Die harmonische Reihe scheint zu konvergieren, was jedoch auf die beschränkte Genauigkeit und das frühe Auftreten von Rundungsfehlern bei \texttt{float16} zurückzuführen ist.
\end{itemize}

\subsubsection{Kahan Summation using float16}

\textbf{Verhalten}
\begin{itemize}
    \item Die Summation zeigt eine stetigere Zunahme im Vergleich zur Vorwärtssummation und konvergiert langsamer.
    \item Der Rundungsfehler wird bei jeder Addition kompensiert, was zu einer genaueren Summation führt.
\end{itemize}

\textbf{Ursache}
\begin{itemize}
    \item Die Kahan-Summation kompensiert die Rundungsfehler durch eine zusätzliche Laufvariable, die den Fehler speichert und in die nächste Addition einbezieht.
\end{itemize}

\textbf{Ergebnis}
\begin{itemize}
    \item Trotz der besseren Genauigkeit gegenüber der Vorwärtssummation ist die Präzision von \texttt{float16} immer noch begrenzt, was zu einer Verzögerung, aber nicht zu einer Verhinderung der scheinbaren Konvergenz führt.
\end{itemize}

\subsubsection{Forward Summation using float32}

\textbf{Verhalten}
\begin{itemize}
    \item Die Summation zeigt zunächst eine stetige Zunahme und erreicht dann einen Punkt, an dem keine signifikante Änderung mehr erfolgt.
    \item Der Konvergenzpunkt tritt später auf als bei \texttt{float16}, aber dennoch deutlich.
\end{itemize}

\textbf{Ursache}
\begin{itemize}
    \item Die höhere Präzision von \texttt{float32} verzögert das Auftreten von Rundungsfehlern, führt jedoch letztendlich zu einem ähnlichen Effekt, bei dem kleine Summanden ignoriert werden.
\end{itemize}

\textbf{Ergebnis}
\begin{itemize}
    \item Die harmonische Reihe scheint auch hier zu konvergieren, was auf die begrenzte Genauigkeit und die später auftretenden Rundungsfehler bei \texttt{float32} zurückzuführen ist.
\end{itemize}

\subsubsection{Kahan Summation using float32}

\textbf{Verhalten}
\begin{itemize}
    \item Die Summation zeigt eine stetige und lineare Zunahme ohne Anzeichen von Konvergenz im beobachteten Bereich.
    \item Die Kahan-Summation kompensiert effektiv die Rundungsfehler und ermöglicht eine genauere Summation über einen längeren Bereich.
\end{itemize}

\textbf{Ursache}
\begin{itemize}
    \item Durch die höhere Präzision von \texttt{float32} und die Kahan-Summation werden die Rundungsfehler wesentlich besser ausgeglichen, wodurch eine scheinbare Konvergenz verhindert wird.
\end{itemize}

\textbf{Ergebnis}
\begin{itemize}
    \item Die harmonische Reihe zeigt keine Anzeichen von Konvergenz im beobachteten Bereich, was die Effektivität der Kahan-Summation bei Verwendung von \texttt{float32} bestätigt.
\end{itemize}

\subsection{Zusammenfassung des Vergleichs}

\begin{itemize}
    \item \textbf{Precision Impact:}
          Die Präzision des Datentyps hat einen signifikanten Einfluss auf die
          Genauigkeit der Summation.
          \texttt{float16} zeigt frühere und stärkere Rundungsfehler als \texttt{float32}.
    \item \textbf{Summation Method Impact:}
          Die Kahan-Summation verbessert die Genauigkeit der Summation erheblich, indem
          sie Rundungsfehler kompensiert, im Gegensatz zur Vorwärtssummation, die
          anfälliger für Fehler ist.
    \item \textbf{Practical Implications:}
          Die Wahl der Summationsmethode und des Datentyps ist entscheidend für
          numerische Berechnungen, insbesondere bei der Handhabung großer Datensätze oder
          langer Reihen, um präzise Ergebnisse zu gewährleisten.
\end{itemize}

\section{Zusammenfassung}
Zusammengefasst kann man sagen, dass der Rechner reelle Zahlen im Binärsystem
darstellt und diese je nach Größe entweder als normalisierte oder subnormale
Gleitkommazahl darstellt.
Abhängig vom gewählten Datentyp können Zahlen genauer dargestellt werden, da
ihnen mehr Bits zur Verfügung stehen.
Außerdem sind diese Maschinenzahlen nicht abgeschlossen bezüglich Addition,
Subtraktion, Multiplikation oder Division, weshalb die Summe, das Produkt,
etc.\ vom Rechner auf die nächstgelegene Gleitkommazahl gerundet werden muss.

Die Auswirkungen auf die harmonische Summe, die bekannterweise divergiert, sind
sehr deutlich.
Die Summanden werden sehr klein und werden für bestimmte Datentypen dann auf
Null abgerundet, was dazu führt, dass die harmonische Reihe für eben diese
Datentypen konvergiert, da ab einem bestimmten Punkt nur noch Nullen addiert
werden.

Die Genauigkeit des Aufsummierens kann im Vergleich zur Vorwärtssummation
erhöht werden, indem man eine andere Summationsmethode verwendet, wie zum
Beispiel die Kahan-Summation.
Dadurch kann das Konvergieren zwar verzögert, aber nicht gestoppt werden, da
die Kahan-Summation von einer Laufvariable abhängt, die den Rundungsfehler
ausgleicht, jedoch nur so genau sein kann wie der jeweilige Datentyp.

\printbibliography

\includepdf[pages=-]{./tools_read_save.pdf}

\end{document}
